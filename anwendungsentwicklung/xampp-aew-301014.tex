%% cheat sheet
% Dokumentformatierung
	% \\				< Zeilenumbruch
	% \par				< Absatz mit Einrückung
	% \\[0.5cm]			< Absatz mit Abstand
	% \newpage			< Neue Seite
	% \part 			< Teil
	% \chapter 			< Kapitel
	% \section 			< Unterüberschrift
	% \subsection 		< Unterunterüberschrift

% Textformatierung
	% \glqq{}foo\{}		< Anführungszeichen
	% \textit{foo}		< Kursiv
	% \textbf{foo}		< Fett
	% \underline{foo}	< Unterstrichsen

% ----------------------------------------------------------------------

% Dokumentenklasse / Definitionen
\documentclass[a4paper,11pt]{scrartcl}	% Format des Dokuments
\usepackage[utf8]{inputenc}				% Interpreter
\usepackage[ngerman] {babel}			% Definition des Sprachraums
\usepackage[T1] {fontenc}				% Schriftart
\usepackage {graphicx}					% Erlaubt das einfügen von Bildern
\usepackage{listings}					% Einfügen von Code

% Kopf- Fußzeile
\usepackage{fancyhdr}
\pagestyle{fancy}
\lhead{Anwendungsentwicklung}
\chead{Thema: XAMPP, PHP Grundlagen}
\rhead{Datum: 30.10.2014}
\lfoot{}
\cfoot{}
\rfoot{Lars Hendrik Friedrichsen - Version 1.1 - \today}


% Header
\title{Beschaffungsablauf}			
\author{Lars Hendrik Friedrichsen}
\date{\today}

% Dokument / Printing 
\begin{document}
\lstset{language=PHP}

\section{Was ist XAMPP?}
XAMPP ist eine freie Apache Distribution und ist ein beliebtes PHP Entwicklungspaket. Es ist leicht zu installieren und
auf allen gängigen Plattformen verfügbar (Windows, Linux u. Mac). \\[0.5cm]
XAMPP steht für:

\begin{itemize}
	\item X: OS
	\item Apache: Server (Webserver)
	\item MySQL: DBMS (Daten-Bank-Management-System)
	\item PHP: Skriptsprache (meist benutzt)
	\item Pearl: Skriptsprache
\end{itemize}

	\subsection{PHP - Grundlegende Funktion}
	PHP ist eine Skriptsprache, die auf dem \textbf{Server} ausgeführt wird. Daher sieht man auch keinen PHP Code, wenn man
	sich im Browser den Code anzeigen lässt. 
	
	\subsection{PHP - Grundlegende Konventionen}
	
	\begin{itemize}
		\item Klassennamen werden \textbf{groß} geschrieben.
		\item Variablen werden mit einem Dollarzeichen (\$) vorgestellt definiert
		\item Variablen werden \textbf{klein} geschrieben
		\item Bei der den Methodiken wird die \glqq Camel Case\grqq \ Schreibweise benutzt. Bsp.: getType
		\item \textbf{Methode:} ist die Funktion einer Klasse
	\end{itemize}
	
	Zu den Konventionen sei gesagt, dass ihnen keine weitere Funktion zukommt. Sie dienen lediglich der Übersichtlichkeit des Codes.
	Außerdem sei hinzugefügt, dass der Parser keine Logkikprüfung vornimmt. Er beachtet lediglich die Syntax
	\footnote{gleichbedeutend mit Rechtschreibung}.
	
	\subsection{Arten der Kommentierung}
	In php gibt es drei Arten der Kommentierung: 
		\begin{lstlisting}[frame=single]	
// foo
# foo
/* foo */
		\end{lstlisting}
	Die ersten beiden Varianten kommentieren immer nur den Code der jeweiligen Zeile aus. Bei der dritten Variante
	kann auch über mehrere Zeilen kommentiert werden.
	
		
\end{document}
