%% cheat sheet
% Dokumentformatierung
	% \\				< Zeilenumbruch
	% \par				< Absatz mit Einrückung
	% \\[0.5cm]			< Absatz mit Abstand
	% \newpage			< Neue Seite
	% \part 			< Teil
	% \chapter 			< Kapitel
	% \section 			< Unterüberschrift
	% \subsection 		< Unterunterüberschrift

% Textformatierung
	% \glqq{}foo\{}		< Anführungszeichen
	% \textit{foo}		< Kursiv
	% \textbf{foo}		< Fett
	% \underline{foo}	< Unterstrichsen

% ----------------------------------------------------------------------

% Dokumentenklasse / Definitionen
\documentclass[a4paper,11pt]{scrartcl}	% Format des Dokuments
\usepackage[utf8]{inputenc}				% Interpreter
\usepackage[ngerman] {babel}			% Definition des Sprachraums
\usepackage[T1] {fontenc}				% Schriftart
\usepackage {graphicx}					% Erlaubt das einfügen von Bildern

% Kopf- Fußzeile
\usepackage{fancyhdr}
\pagestyle{fancy}
\lhead{LF 8 / Saß}
\chead{Thema: Angebote}
\rhead{Datum: 12.11.2014}
\lfoot{Lars Hendrik Friedrichsen - Version 1.0 - \today}
\cfoot{}
\rfoot{\thepage}


% Header
\title{Die Anfragae}			
\author{Lars Friedrichsen}
\date{\today}

% Dokument / Printing 
\begin{document}

\section{Das verbindliche Angebot}

\textbf{Begriff:} Die Bindung betrifft immer den Verkäufer.Der Käufer ist grundsätzlich nicht gebunden
	
	\begin{itemize}
		\item Der Verkäufer muss sich an alle Angaben des Angebotes halten
		\item Es ist keine Änderung möglich
	\end{itemize}
	
	\subsection{Schriftliche Angebote}
	\begin{itemize}
		\item Auch Angebote unter Abwesenden genannt (Fax)
		\item Kunde muss umgehend bestellen, d.h.:\\
		Beförderungszeit + Überlegungszeit + Rückbeförderung = 7 Tage (gesetzl. Regelung)
	\end{itemize}
	
	\subsection{Mündliche Angebote}
	\begin{itemize}
		\item Auch Angebot unter Anwesenden genannt
		\item Müssen während des Gesprächs angenommen werden
	\end{itemize}
	\textbf{Merke:} Widerruf möglich. Der Widerruf muss jedoch gleichzeitig mit dem fehlerhaften Angebot
	beim Kunden eintreffen. Bei einer abgeänderten Bestellung verliert das Angebot seine Gültigkeit.

\section{Das unverbindliche Angebot}

\begin{itemize}
	\item Grund: Der Verkäufer möchte sein Risiko minimieren, deshalb \underline{Freizeichnungsklauseln}
		\begin{itemize}
			\item Preise freibleibend
			\item Preisänderung vorbehalten
			\item Lieferung solange Vorrat reicht
			\item \dots bieten wir ihnen freibleibend an
		\end{itemize}
	\item Die Bindung wird komplett aufgehoben
	\begin{itemize}
		\item Angebot freibleibend
		\item Angebot unverbindlich
		\item \dots ohne Obligo
	\end{itemize}
\end{itemize}

\subsection{Befristete Angebote}

\begin{itemize}
	\item Der Verkäufer gibt den Zeitpunkt im Angebot an, danach hat das Angebot keine Gültigkeit mehr.
\end{itemize}

\subsection{Schaufensterauslagen / Prospekte}
\begin{itemize}
	\item Sind Aufforderungen zum Kauf und keine Angebote an eine bestimmte Person.
	\item Es entsteht keine rechtliche Verpflichtung an das Angebot
\end{itemize}

\end{document}
	
