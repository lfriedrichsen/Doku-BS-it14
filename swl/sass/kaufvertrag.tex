%% cheat sheet
% Dokumentformatierung
	% \\				< Zeilenumbruch
	% \par				< Absatz mit Einrückung
	% \\[0.5cm]			< Absatz mit Abstand
	% \newpage			< Neue Seite
	% \part 			< Teil
	% \chapter 			< Kapitel
	% \section 			< Unterüberschrift
	% \subsection 		< Unterunterüberschrift

% Textformatierung
	% \glqq{}foo\{}		< Anführungszeichen
	% \textit{foo}		< Kursiv
	% \textbf{foo}		< Fett
	% \underline{foo}	< Unterstrichsen

% ----------------------------------------------------------------------

% Dokumentenklasse / Definitionen
\documentclass[a4paper,11pt]{scrartcl}	% Format des Dokuments
\usepackage[utf8]{inputenc}				% Interpreter
\usepackage[ngerman] {babel}			% Definition des Sprachraums
\usepackage[T1] {fontenc}				% Schriftart
\usepackage {graphicx}					% Erlaubt das einfügen von Bildern

% Kopf- Fußzeile
\usepackage{fancyhdr}
\pagestyle{fancy}
\lhead{LF 8 / Saß}
\chead{Thema: Die Anfrage}
\rhead{Datum: 14.11.2014}
\lfoot{\thepage}
\cfoot{}
\rfoot{Lars Hendrik Friedrichsen - Version 1.0 - \today}


% Header
\title{Die Anfragae}			
\author{Lars Friedrichsen}
\date{\today}

% Dokument / Printing 
\begin{document}

\section{Der Kaufvertrag}

	\subsection{Willenserklärugen}
	Damit ein Kaufvertrag zu Stande kommt, bedarf es zweier übereinstimmender WEs\footnote{Willenserklärung}
	Die WEs sind wie folgt benannt:
	
	\begin{enumerate}
		\item \textbf{Antrag:} Der Antrag kann sowohl vom Käufer als auch vom Lieferanten kommen.
			Es ist situationsabhängig. Eine einfahe Bestellung ist ein Antrag. Ein verbindliches 
			Angebot ist auch ein Antrag.
		\item \textbf{Annahme:} Die Annahme kann sowohl  vom Käufer als auch vom Lieferer kommen. Eine Bestellung auf Grund
			eines verbindlichen Angebotes ist eine Annahme. Das versenden einer Auftragsbestätigung, auf Grund 
			einer einfachen Bestellung ist auch eine Annahme.
	\end{enumerate}

	\subsection{Rechte und Pflichten aus dem Kaufvertarg}

	\begin{itemize}
		\item \textbf{Pflichten des Verkäufers:}
			\begin{enumerate}
				\item Der Verkäufer muss die Ware zu den vereinbarten Bedingugngen liefern.
				\item Der Verkäufer muss die Eigentumsübertragung ermöglichen.
			\end{enumerate}
		\item \textbf{Pflichten des Käufers:}
			\begin{enumerate}
				\item Der Käufer ist verpflichtet die Ware anzunehmen und die Eigentumsübertragung
					nicht zu behindern.
				\item Der Käufer ist verpflichtet den Kaufvertrag wie vereinbart zu erfüllen (Zahlung)
			\end{enumerate}
	\end{itemize}

\newpage

\section{Übungsaufgabe --  Lückentext}
Ein Kaufvertrag kommt durch zwei \underline{übereinstimmende} Willeenserklärungen zustande. Die zuerst abgegebene
Willenserklärung nennt man \underline{Antrag}, die zweite \underline{Annahme}. Der \underline{Antrag} zum Abschluss eines
Kaufvertrags kann vom \underline{Verkäufer} oder vom \underline{Käufer} ausgehen. Das \underline{Angebot} ist der 
Antrag des Verkäufers. Nimmt der Käufer das \underline{Angebot} durch eine \underline{Bestellung} an, so ist der Kaufvertrag
\underline{abgeschlossen}.\\[0.5cm]
Bestellt der Käufer ohne vorheriges \underline{Angebot}, so ist diese Bestellung ein \underline{Angebot}
des Käufers auf den Abschluss eines kaufvertrages. Bestätigt der Verkäufer die Bestellung durch eine \underline{Auftragsbestätigung},
so ist der Kaufvertrag \underline{zustanden gekommen}.\\[0.5cm]
Durch den Abschluss eines Kaufvertrags ubernehmen der Verkäufer under Käufer \underline{Pfllichten}. Der Verkäufer
\underline{verpflichtet} sich, die Ware \underline{rechtzeitig} und \underline{einwandfrei} zu liefern und dem Käufer
das \underline{Eigentum} zu \underline{übertragen}. Der Käufer \underline{verpflichtet} sich, den \underline{verinbarten}
Kaufpreis \underline{rechtzeitig} zu zahlen und die Ware \underline{anzunehmen}.\\[0.5cm]
Den Abschluss eines Kaufvertrages bezeichnet man daher auch als \underline{Verpflichtungsgeschäft}.
Bei der Erfüllung des Kaufvertrages erfüllen der Verkäufer und der Käufer ihre Pflichten \underline{vereinbarungsgemäß}.
Man spricht hier vom \underline{Erfüllungsgeschäft}.
\end{document}
	
