%% cheat sheet
% Dokumentformatierung
	% \\				< Zeilenumbruch
	% \par				< Absatz mit Einrückung
	% \\[0.5cm]			< Absatz mit Abstand
	% \newpage			< Neue Seite
	% \part 			< Teil
	% \chapter 			< Kapitel
	% \section 			< Unterüberschrift
	% \subsection 		< Unterunterüberschrift

% Textformatierung
	% \glqq{}foo\{}		< Anführungszeichen
	% \textit{foo}		< Kursiv
	% \textbf{foo}		< Fett
	% \underline{foo}	< Unterstrichsen

% ----------------------------------------------------------------------

% Dokumentenklasse / Definitionen
\documentclass[a4paper,11pt]{scrartcl}	% Format des Dokuments
\usepackage[utf8]{inputenc}				% Interpreter
\usepackage[ngerman] {babel}			% Definition des Sprachraums
\usepackage[T1] {fontenc}				% Schriftart
\usepackage {graphicx}					% Erlaubt das einfügen von Bildern

% Kopf- Fußzeile
\usepackage{fancyhdr}
\pagestyle{fancy}
\lhead{LF 8 / Saß}
\chead{Thema: Beschaffungsmarketing}
\rhead{Datum: 03.11.2014}
\lfoot{\thepage}
\cfoot{}
\rfoot{Lars Hendrik Friedrichsen - Version 1.0 - \today}


% Header
\title{Beschaffungsmarketing}			
\author{Lars Friedrichsen}
\date{\today}

% Dokument / Printing 
\begin{document}

\section{Beschaffungsmarketing}


\section{Aufgaben:}

	\begin{enumerate}
		\item Definiere den Begriff Beschaffungsmarketing!
		\item Nenne und erläutere die einzelnen Quellen (Beschaffungsteilmärkte) des Beschaffungsmarketings!
		\item Wodurch unterscheidet sich Beschaffungsmarketing und Güterbeschaffungsmarketing?
		\item Erläutern Sie, weshalb der Absatzplan einer Unternehmung Grundlage des Beschaffungsmarketings ist!
		\item Welche Fragen sollen im Rahmen des Beschaffungsmarketings geklärt werden?
	\end{enumerate}
	
	\subsection{Lösungen:}
	
		\begin{enumerate}
			\item Zum Beschaffungsmarketing gehören im weitesten Sinne alle Tätigkeiten, die sich auf die Beschaffung
			und termingerechte Bereitstellung der betrieblichen Produktionsfaktoren beziehen.
			Zum Ziel des Beschaffungsmarketings gehört es, langfristig u. nachhaltig Bezugsquellen aufzubauen und zu halten.
			\item Das Beschaffungsmarketing unterteilt sich in folgende Unterbereiche:
				\begin{itemize}
					\item \textbf{Arbeitskräfte}
					\item \textbf{Finanzmittel} 
					\item \textbf{Betriebsmittel} 
					\item \textbf{Handelswaren}
					\item \textbf{Informationen}
				\end{itemize}				 
			\item 
			\item 
			\item
		\end{enumerate}
	
	

\end{document}
	
