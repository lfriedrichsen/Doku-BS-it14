%% cheat sheet
% Dokumentformatierung
	% \\				< Zeilenumbruch
	% \par				< Absatz mit Einrückung
	% \\[0.5cm]			< Absatz mit Abstand
	% \newpage			< Neue Seite
	% \part 			< Teil
	% \chapter 			< Kapitel
	% \section 			< Unterüberschrift
	% \subsection 		< Unterunterüberschrift

% Textformatierung
	% \glqq{}foo\{}		< Anführungszeichen
	% \textit{foo}		< Kursiv
	% \textbf{foo}		< Fett
	% \underline{foo}	< Unterstrichsen

% ----------------------------------------------------------------------

% Dokumentenklasse / Definitionen
\documentclass[a4paper,11pt]{scrartcl}	% Format des Dokuments
\usepackage[utf8]{inputenc}				% Interpreter
\usepackage[ngerman] {babel}			% Definition des Sprachraums
\usepackage[T1] {fontenc}				% Schriftart
\usepackage {graphicx}					% Erlaubt das einfügen von Bildern

% Kopf- Fußzeile
\usepackage{fancyhdr}
\pagestyle{fancy}
\lhead{LF 8 / Saß}
\chead{Thema: ABC-Analyse}
\rhead{Datum: 05.11.2014}
\lfoot{\thepage}
\cfoot{}
\rfoot{Lars Hendrik Friedrichsen - Version 1.0 - \today}


% Header
\title{ABC-Analyse}			
\author{Lars Friedrichsen}
\date{\today}

% Dokument / Printing 
\begin{document}

\section{Die ABC-Analyse}
Unter der ABC-Analyse versteht man die wertmäßige Gruppierung von zu beschaffenden Waren.
Sie dient als Hilfsmittel bei der Optimierung der Warenbeschaffung. Je nach Einstufung kommt den Gütern eine
andere Aufmerksamkeit zu. Der A-Gruppe kommt dem größten Kontrollbedarf zu.

\section{Arbeitsaufträge}

\begin{enumerate}
	\item \textbf{Beschreiben Sie die Arbeitsschritte der ABC-Analyse!}\par
		\begin{enumerate}
			\item Ermittlung des Beschaffungswertes
			\item Absteigende Sortierung nach Warenwert
			\item Kumulation der Werte
			\item Einteilung in die jeweiligen Gruppen
		\end{enumerate}
	\item \textbf{Was sind A-, B- und C-Gruppen?}\par
	Unter den Gruppen versteht man die anteilige Wertigkeit der Güter zum Gesamtwert. Die Gruppen sind
	in \textbf{A-, B- u. C-Gruppe} unterteilt. Die Verteilung nach Werten sieht wie folgt aus:
		\begin{itemize}
			\item \textbf{A-Materialien:} Diese Gruppe hat etwa einen Anteil von \textbf{65-80\%} des Gesamtwertes
			\item \textbf{B-Materialien:} Diese Gruppe hat etwa einen Anteil von \textbf{15-20\%} des Gesamtwertes
			\item \textbf{C-Materialien:} Diese Gruppe hat etwa einen Anteil von \textbf{5-10\%} des Gesamtwertes
		\end{itemize}
	\item \textbf{Welche Konsequenzen zieht die Materialbeschaffung aus dieser Einteilung?}\par
	Durch das Verwenden der ABC-Analyse können Schwächen in der Lagerhaltung und im Beschaffungsmarketing
	aufgedeckt werden. Außerdem kann nun ein Fokus auf die anteilig wertmäßig größten Güter gelegt werden, da hier 
	eine kleine Änderung (im Preis) sich schon signifikant auf den Erfolg auswirkt.
\end{enumerate}

\end{document}
	
