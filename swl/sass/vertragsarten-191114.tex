%% cheat sheet
% Dokumentformatierung
	% \\				< Zeilenumbruch
	% \par				< Absatz mit Einrückung
	% \\[0.5cm]			< Absatz mit Abstand
	% \newpage			< Neue Seite
	% \part 			< Teil
	% \chapter 			< Kapitel
	% \section 			< Unterüberschrift
	% \subsection 		< Unterunterüberschrift

% Textformatierung
	% \glqq{}foo\{}		< Anführungszeichen
	% \textit{foo}		< Kursiv
	% \textbf{foo}		< Fett
	% \underline{foo}	< Unterstrichsen

% ----------------------------------------------------------------------

% Dokumentenklasse / Definitionen
\documentclass[a4paper,11pt]{scrartcl}	% Format des Dokuments
\usepackage[utf8]{inputenc}				% Interpreter
\usepackage[ngerman] {babel}			% Definition des Sprachraums
\usepackage[T1] {fontenc}				% Schriftart
\usepackage {graphicx}					% Erlaubt das einfügen von Bildern

% Kopf- Fußzeile
\usepackage{fancyhdr}
\pagestyle{fancy}
\lhead{LF 8 / Saß}
\chead{Thema: Vertragsarten}
\rhead{Datum: 19.11.2014}
\lfoot{\thepage}
\cfoot{}
\rfoot{Lars Hendrik Friedrichsen - Version 1.0 - \today}


% Header
%\title{}			
\author{Lars Friedrichsen}
\date{\today}

% Dokument / Printing 
\begin{document}

\section{Vertragsarten}

Käufer und Verkäufer können nach ihrer Rechtsstellung Kaufleute oder Nichtkaufleute sein.

	\begin{enumerate}
		\item \textbf{Bürgerlicher Kauf:}
			\begin{itemize}
				\item beide Parteien sind Privatpersonen
				\item es liegt kein Handelsgeschäft vor
				\item Bsp. Lebensmittelhändler kauft von seinem Spediteur eine gebrauchte
					Taucherausrüstung
			\end{itemize}
		\item \textbf{Handelskauf:}
			\begin{itemize}
				\item \textbf{zweiseitiger Handelskauf}
					\begin{itemize}
						\item beide Parteien sind Kaufleute
						\item Kaufvertrag wird für geschäftliche Zwecke abgeschlossen
						\item Bsp. FSG kauft Stahlplatten bei einem Walzwerk
					\end{itemize}
				\item \textbf{einseitiger Handelskauf:}
					\begin{itemize}
						\item Eine Partei ist Kaufmann, die andere nicht
					\end{itemize}
			\end{itemize}
	\end{enumerate}
		
\section{}


\end{document}
	
