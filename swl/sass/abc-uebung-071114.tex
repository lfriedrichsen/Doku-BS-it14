%% cheat sheet
% Dokumentformatierung
	% \\				< Zeilenumbruch
	% \par				< Absatz mit Einrückung
	% \\[0.5cm]			< Absatz mit Abstand
	% \newpage			< Neue Seite
	% \part 			< Teil
	% \chapter 			< Kapitel
	% \section 			< Unterüberschrift
	% \subsection 		< Unterunterüberschrift

% Textformatierung
	% \glqq{}foo\{}		< Anführungszeichen
	% \textit{foo}		< Kursiv
	% \textbf{foo}		< Fett
	% \underline{foo}	< Unterstrichsen

% ----------------------------------------------------------------------

% Dokumentenklasse / Definitionen
\documentclass[a4paper,11pt]{scrartcl}	% Format des Dokuments
\usepackage[utf8]{inputenc}				% Interpreter
\usepackage[ngerman] {babel}			% Definition des Sprachraums
\usepackage[T1] {fontenc}				% Schriftart
\usepackage {graphicx}					% Erlaubt das einfügen von Bildern

% Kopf- Fußzeile
\usepackage{fancyhdr}
\pagestyle{fancy}
\lhead{LF 8 / Saß}
\chead{Thema: ABC-Analyse}
\rhead{Datum: 07.11.2014}
\lfoot{\thepage}
\cfoot{}
\rfoot{Lars Hendrik Friedrichsen - Version 1.0 - \today}


% Header
\title{ABC-Analyse}			
\author{Lars Friedrichsen}
\date{\today}

% Dokument / Printing 
\begin{document}

\section{Übungsaufgabe}

	\begin{enumerate}
		\item \textbf{Begründen Sie die Notwendigkeit einer ABC-Analyse bei der Beschaffung von Materialien!}\par
		Die ABC-Analyse ist erforderlich, um einen wertmäßigen Eindruck der zu beschaffenden Materialien zu bekommen.
		Nur so kann die Beschaffung der verschiedenen Materialien optimiert werden.  
		\item \textbf{Erstellen Sie anhand obiger Angaben eine ABC-Analyse und werten sie diese tabellarisch und grafisch aus!}
		\item \textbf{Entscheiden und begründen Sie für jedes Material, ob eine genaue Bedarfsermittlung oder eine weniger
		genaue Bedarfsermittlung / Schätzung erforderlich ist!}\par
		Bei der \textbf{A-Gruppe} ist auf jeden Fall eine genau Bedarfsermittlung erforderlich, da diese schon
		über die Hälfte, und mit den Gasfedern schon knappe 80\% ausmachen. Hier hat eine Optimierung direkten
		Einfluss auf den Erfolg.
		Bei der \textbf{B-Gruppe} ist eine Abwägung nötig, da hier nicht allzu hohe Werte vorhanden sind. Jedoch 
		kann auch in dieser Gruppe Bedarf bestehen einen Artikel genauer zu betrachten. Bei geringwertigeren 
		Materialien reicht eine Schätzung aus, da der Aufwand den Nutzen ansonsten übersteigt.   
		\item  \textbf{Berechnen Sie für den Monat Januar den voraussichtlichen Bedarf an Schrauben M6 mithilfe
		der Methode der Durchschnittswerte}\par
			\begin{tabular}{|l|l|l|l|l|l|}
				\hline
				Juli	&	6000	&	September	&	8000	&	November	&	8800\\ \hline
				August	&	7000	&	Oktober		&	6800	&	Dezember	&	9000\\ \hline
			\end{tabular}\\[0.5cm]
			
		\item \textbf{Ermitteln Sie den Materialbedarf an Schrauben M6 für Januar mithilfe des trendkorrigierten
		gewogenen Durchschnitts, wenn folgende Gewichtung zugrunde gelegt wird: Juli 6\%, August 8\%, September 15\%,
		Oktober 18\%, November 23\%, Dezember 30\%!}
	\end{enumerate}



\end{document}
	
