%% cheat sheet
% Dokumentformatierung
	% \\				< Zeilenumbruch
	% \par				< Absatz mit Einrückung
	% \\[0.5cm]			< Absatz mit Abstand
	% \newpage			< Neue Seite
	% \part 			< Teil
	% \chapter 			< Kapitel
	% \section 			< Unterüberschrift
	% \subsection 			< Unterunterüberschrift

% Textformatierung
	% \glqq{}foo\{}		< Anführungszeichen
	% \textit{foo}		< Kursiv
	% \textbf{foo}		< Fett
	% \underline{foo}	< Unterstrichsen

% ----------------------------------------------------------------------

% Dokumentenklasse / Definitionen
\documentclass[a4paper,11pt]{scrartcl}		% Format des Dokuments
\usepackage[utf8]{inputenc}			% Interpreter
\usepackage[ngerman] {babel}			% Definition des Sprachraums
\usepackage[T1] {fontenc}			% Schriftart
\usepackage {graphicx}				% Erlaubt das einfügen von Bildern

% Kopf- Fußzeile
\usepackage{fancyhdr}
\pagestyle{fancy}
\lhead{LF 1 / Hofmann}
\chead{Thema: Rationaliserungen}
\rhead{Datum: 12.11.2014}
\lfoot{\thepage}
\cfoot{}
\rfoot{Lars Hendrik Friedrichsen - Version 1.0 - \today}


% Header
\title{Der Betrieb und sein Umfeld}			
\author{Lars Friedrichsen}
\date{\today}

% Dokument / Printing 
\begin{document}

\section{Rationalisierungen}

\begin{enumerate}
	\item Der technische Fortschritt und die damit verbundene Rationalisierung führte dazu, dass
		eine Reihe von Berufen in der Vergagenheit stark zurückgedrängt wurden oder ganz verloren gingen.\\
		\textbf{Kennen sie folgende Berufe?}\par
		\begin{tabular}{|l|l|}
			\hline
			Beruf			&	Kurze Beschreibung\\ \hline
			Schriftsetzer		&	Erstellungn von Druckvorlagen, Bsp. für den Buchdruck\\ \hline
			Böttcher		&	Stellen Fässer her\\ \hline
			Köhler			&	Stellen Kohle her\\ \hline
			Harzer			&	Harzgewinnung\\ \hline
			Haderlump		&	Sammeln von alter Kleidung und weiterverkauf an z.B. Mühlen\\ \hline
			Rademacher		&	Stellt komplizerte Räder her\\ \hline
		\end{tabular}
	\item Rationalisierung führt zum Prozess der \glqq schöpferische Zerstörung\grqq. Geben Sie hierzu zwei Beispiele aus
		der Wirtschaftspraxis an, welche Arbeitsprozesse (oder Berufe) in der Vergangenheit \glqq zersört\grqq wurden
		und welche neu geschaffen wurden.\par
		\begin{enumerate}
			\item Schmied
			\item Müller
			\item IT-Berufe
			\item Gamer
		\end{enumerate}
	\item Welche Berufe sind zukunftsfähig, welche Berufe werden durch Rationalisierung in den nächsten Jahren an
		Bedeutung verlieren?\par
		Zukunftsfähig sind Berufe, die eine gewisse Qualifikation erfordern und eine geistige Tätigkeit vorraussetzen.
	\item welche Maßnahmen nkann ein Arbeitnehmer ergreifen, um sich gegen Rationalisierung zu schätzen?\par
		Arbeitnhemer sind stets angehalten sich weiterzubilden um der Rationalisierung zu entkommen.
\end{enumerate}

\end{document}
	
