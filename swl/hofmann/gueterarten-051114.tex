%% cheat sheet
% Dokumentformatierung
	% \\				< Zeilenumbruch
	% \par				< Absatz mit Einrückung
	% \\[0.5cm]			< Absatz mit Abstand
	% \newpage			< Neue Seite
	% \part 			< Teil
	% \chapter 			< Kapitel
	% \section 			< Unterüberschrift
	% \subsection 		< Unterunterüberschrift

% Textformatierung
	% \glqq{}foo\{}		< Anführungszeichen
	% \textit{foo}		< Kursiv
	% \textbf{foo}		< Fett
	% \underline{foo}	< Unterstrichsen

% ----------------------------------------------------------------------

% Dokumentenklasse / Definitionen
\documentclass[a4paper,11pt]{scrartcl}	% Format des Dokuments
\usepackage[utf8]{inputenc}				% Interpreter
\usepackage[ngerman] {babel}			% Definition des Sprachraums
\usepackage[T1] {fontenc}				% Schriftart
\usepackage {graphicx}					% Erlaubt das einfügen von Bildern

% Kopf- Fußzeile
\usepackage{fancyhdr}
\pagestyle{fancy}
\lhead{LF 1 / Hofmann}
\chead{Thema: Der Betrieb und sein Umfeld}
\rhead{Datum: 05.11.2014}
\lfoot{\thepage}
\cfoot{}
\rfoot{Lars Hendrik Friedrichsen - Version 1.0 - \today}


% Header
\title{Der Betrieb und sein Umfeld}			
\author{Lars Friedrichsen}
\date{\today}

% Dokument / Printing 
\begin{document}

\section{Güterarten}
Es wird zwischen verschieden Güterarten unterschieden, welche im folgenden dargelegt werden.

\section{Aufgaben}

\begin{enumerate}
\item \textbf{Recherchieren Sie folgende Begriffe:}
	
	\begin{itemize}
		\item \textbf{Substitutionsgüter:} Güter, die durch andere Güter \textbf{ersetzt} werden, z.B.: Butter <> Margarine
		\item \textbf{Komplementärgüter:} Güter, die durch andere Güter \textbf{ergänzt} werden, z.B.: Auto -> Benzin 
		\item \textbf{inferiore Güter:} Güter, die bei steigendem Einkommen weniger nachgefragt werden (einkommensabhängig)
		z.B.: Auto, Kleidung
		\item \textbf{superiore Güter:} Güter, die bei steigendem Einkommen überproportional teurer sind als die vorherigen.
	\end{itemize}
	
\item \textbf{Unter welchen Bedingungen ist ein Notebook ein:}
	\begin{itemize}
		\item \textbf{Konsumgut?:}\par
		Das Notebook ist ein Konsumgut, wenn es für den privaten Zweck verwendet wird.
		\item \textbf{Investitions / Produktionsgut?:}\par
		Das Notebook ist ein Investitions- / Produktionsgut, wenn es im Betrieb zur Produktion oder zur Arbeitsverrichtung
		eingesetzt wird.
	\end{itemize}
	
\item \textbf{Was sind Produktionsfaktoren?}
	\begin{itemize}
		\item Welche werden in der VWL unterschieden?\par
		In der VWL unterscheidet man zwischen folgenden Produktionsfaktoren:
			\begin{itemize}
				\item \textbf{Arbeit:} Die von den Menschen zur Verfügung gestellte Komponente (geistig wie physisch)
				\item \textbf{Kapital:} Die zur Produktion notwendigen Güter, z.B.: Maschinen, Anlagen, Werkstoffe.
				Diese werden auch als \textbf{Realkapital} bezeichnet.
				\item \textbf{Natur:} Hierunter fallen z.B.: Rohstoffvorkommen und die geografische Begrenzung eines Landes.
			\end{itemize}
		\item Welche in der BWL?\par
		In der BWL unterscheidet man zwischen den folgenden Bereichen:
		\begin{itemize}
		\item \textbf{Elementarfaktoren:}
					
			\begin{itemize}
				\item \textbf{Arbeitsleistung:} Die Komponente die benötigt wird, um die Arbeit zielgerecht zu verrichten.
				\item \textbf{Betriebsmittel:} Die Komponenten die zur Produktion benötigt werden, aber nicht in das
				Produkt einfließen, z.B.: Maschinen, Gebäude
				\item \textbf{Werkstoffe:} Die Komponenten die zur Herstellung der Erzeugnisse benötigt werden, 
				z.B.: Stahl, Holz
			\end{itemize}
		\item \textbf{Dispositve Faktoren:}
			\begin{itemize}
				\item \textbf{leitende Aufgaben:} Hat eine weisungsbefugte Rolle
				\item \textbf{Organisation / Aufgabenverteilung:} Nimmt die Verteilung und Verwaltung der Arbeit vor.
				\item \textbf{Kontrolle:} Stellt fest, ob die Arbeit ordentlich verrichtet wurde.
			\end{itemize}
		\end{itemize}
	\end{itemize}
\end{enumerate}	


\end{document}
	
