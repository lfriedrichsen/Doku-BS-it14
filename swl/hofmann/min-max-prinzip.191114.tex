%% cheat sheet
% Dokumentformatierung
	% \\				< Zeilenumbruch
	% \par				< Absatz mit Einrückung
	% \\[0.5cm]			< Absatz mit Abstand
	% \newpage			< Neue Seite
	% \part 			< Teil
	% \chapter 			< Kapitel
	% \section 			< Unterüberschrift
	% \subsection 			< Unterunterüberschrift

% Textformatierung
	% \glqq{}foo\{}		< Anführungszeichen
	% \textit{foo}		< Kursiv
	% \textbf{foo}		< Fett
	% \underline{foo}	< Unterstrichsen

% ----------------------------------------------------------------------

% Dokumentenklasse / Definitionen
\documentclass[a4paper,11pt]{scrartcl}		% Format des Dokuments
\usepackage[utf8]{inputenc}			% Interpreter
\usepackage[ngerman] {babel}			% Definition des Sprachraums
\usepackage[T1] {fontenc}			% Schriftart
\usepackage {graphicx}				% Erlaubt das einfügen von Bildern

% Kopf- Fußzeile
\usepackage{fancyhdr}
\pagestyle{fancy}
\lhead{LF 1 / Hofmann}
\chead{Thema: Das ökonomische Prinzip}
\rhead{Datum: 19.11.2014}
\lfoot{\thepage}
\cfoot{}
\rfoot{Lars Hendrik Friedrichsen - Version 1.0 - \today}


% Header
%\title{}			
\author{Lars Friedrichsen}
\date{\today}

% Dokument / Printing 
\begin{document}

\section{Themen für die Klausur}

	\begin{itemize}
		\item Bedürfnisse / Bedraf
		\item Güterarten
		\item Produktionsfaktoren
		\item ökonomisches Prinzip
		\item Shareconomy
	\end{itemize}

\section{Ökonomisches Prinzip}

	\subsection{Aufgaben zum Arbeitsblatt}

	\textbf{\underline{Aufgabe 1:} Beschreiben Sie die Zielsetzungen, die die einzelnen Personen verfolgen!}

		\begin{itemize}
			\item Verbesserung des Image (UL)
			\item Hohe Qualität der Anlage (Hr. Jordan)
			\item Ausgaben begrenzen (Fr. Treiber)
		\end{itemize}

	\textbf{\underline{Aufgabe 2:} Welche Personen handeln nach dem Maximalprinzip?}\par
	Frau Treiber verfolgt das Maximalprinzip, da sie die Mittel begrenzt hat und damit das Beste erreichen möchte.\\[0.5cm]

	\textbf{\underline{Aufgabe 3} Welche Personen wenden das Minimalprinzip an?}\par
	Die UL, da das Ziel festgelegt ist und so wenig wie möglich investiert werden soll. Außerdem folgt er Jordan dem 
	Minimalprinzip, da er gute Qualität zu einem günstigen Preis möchte.

	\subsection{Geben sie an, ob es sich um das Minimum- oder Maximumprinzip zum tragen kommt (Buch S. 37)}

		\begin{enumerate}
			\item Für den neuen MP3 Player soll ein bestimmter Marktanteil mit einem geringen Werbeetat erreicht
				werden:\par
				\textbf{Minimalrinzip}
			\item Die LKWs sollen so viele Kunden wie möglich beliefern:\par
				\textbf{Maximalprinzip}
			\item Für die Ausstattung eines Mainboards werden die günstigsten Speicherbausteine verwendet:\par
				\textbf{Minimalrinzip}
			\item Die Mitarbeiter überlegen, ob sie für das Aufsetzen von 100 PCs den Aufwand so gering wie möglich
				halten sollen:\par
				\textbf{Minimalrinzip}
			\item Bei der Vernetzung sollen möglichst wenig Kabel gelegt werden:\par
				\textbf{Minimalrinzip}
			\item Die WG will mit einem monatlichen Haushaltsgeld von EUR 550 auskommen:\par
				\textbf{Maximalprinzip}
			\item Der Azubi möchte möglichst viel Fachliteratur für EUR 20 monatlich bekommen:\par
				\textbf{Maximalprinzip}
			\item Wegen der Inventur soll die Schließungszeit möglichst gering sein:\par
				\textbf{Minimalrinzip}
		\end{enumerate}
\end{document}
	
