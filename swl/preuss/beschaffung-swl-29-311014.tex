%% cheat sheet
% Dokumentformatierung
	% \\					< Zeilenumbruch
	% \par				< Absatz mit Einrückung
	% \\[0.5cm]			< Absatz mit Abstand
	% \newpage			< Neue Seite
	% \part 				< Teil
	% \chapter 			< Kapitel
	% \section 			< Unterüberschrift
	% \subsection 		< Unterunterüberschrift

% Textformatierung
	% \glqq{}foo\{}		< Anführungszeichen
	% \textit{foo}		< Kursiv
	% \textbf{foo}		< Fett
	% \underline{foo}	< Unterstrichsen

% ----------------------------------------------------------------------

% Dokumentenklasse / Definitionen
\documentclass[a4paper,11pt]{scrartcl}	% Format des Dokuments
\usepackage[utf8]{inputenc}				% Interpreter
\usepackage[ngerman] {babel}			% Definition des Sprachraums
\usepackage[T1] {fontenc}				% Schriftart
\usepackage {graphicx}					% Erlaubt das einfügen von Bildern

% Kopf- Fußzeile
\usepackage{fancyhdr}
\pagestyle{fancy}
\lhead{BWP/LF 8/Pr}
\chead{Thema: Beschaffung}
\rhead{Datum: 29.10.2014}
\lfoot{\thepage}
\cfoot{}
\rfoot{Lars Hendrik Friedrichsen - Version 1.1 - \today}


% Header
\title{Beschaffungsablauf}			
\author{Lars Friedrichsen}
\date{\today}

% Dokument / Printing 
\begin{document}

\section{Situation (Zusammenfassung)}

Die Stadt Flensburg beauftragt die \glqq Otto Dattelmann KG\grqq 150 Pcs und 200 Monitore zu liefern.
Die OD KG hat die Ware nicht am Lager und muss bestellen. \\[0.5cm]
Wie sieht der Beschaffungsvorgang aus?
	\subsection{Allgemeine Möglichkeiten für die Warenbeschaffung}

		\begin{enumerate}
			\item bekannte Lieferanten
			\item neue Lieferanten
			\item Neubestellung nicht gelisteter Ware bei bekannten / unbekannten Lieferanten
		\end{enumerate}
		
	\subsection{Bezugsquellenermittlung}
	
		\begin{tabular}{|l|l|}
			\hline
			\textbf{interne Bezugsquellen}			&	\textbf{externe Bezugsquellen}\\ \hline
			bekannter Lieferant				&	Internetrecherche \\ \hline
			eigene Erfahrung / Mitarbeiter	&	Empfehlung von Dienstleister(Branchenbuch, gelbe Seiten)\\ \hline
			alte Angebote					&	Wirtschaftsverband \\ \hline
			Besuch von Vertretern			&	Geschäftsberichte \\ \hline
			Messebesuche						&	\\ \hline
		\end{tabular}

	\subsection{Die Anfrage}
	
		\begin{enumerate}
			\item \textbf{allgemeine Anfrage:} Bsp.: Preisliste Prospekte
			\item \textbf{bestimmte Anfrage:} Bsp.: Bestimmtes Produkt / Menge / Qualität / Termin
		\end{enumerate}
		
	\textbf{Rechtliche Bindung:} Eine Anfrage hat \textbf{keine} rechtliche Bindung und iat demnach unverbindlich. Der potentielle Käufer ist nicht verpflichtet die Ware zu kaufen.\\[0.5cm]
	
	\subsection{Aufgaben}
	
		\begin{enumerate}
			\item Warum müssen Unternehmen mit festen Lieferanten immer auch bei möglichen neuen Informationsquellen anfragen? \par
			
			\item Nennen Sie Vor- und Nachteile einer telefonischen Anfrage. \par
			
			\item Warum richtet ein Unternehmen, dass ein Sortiment erweitern will, Anfragen an verschiedene Lieferanten? \par
			
		\end{enumerate}
\section{Beschaffungsplanung}
 	
 	\subsection{Was soll bestellt werden?}
 		
 		\begin{itemize}
 			\item \textbf{Ziele:} bedarfsorientiertes Sortiment
 			\item \textbf{Sortimentsplanung} abhängig von: Bedarf, Kundenwunsch, Art, Güte und Beschaffenheit, Lieferbereitschaft, Geschäftszweig, 
 		\end{itemize}
 		
 	\subsection{Wie viel soll bestellt werden?}
 	
 		\begin{itemize}
 			\item \textbf{Ziele:} optimale Bestellmenge
 			\item \textbf{Mengenplanung} abhängig von: Lagerkapazität, Haltbarkeit, Durchschnittsverbrauch, Saison
 		\end{itemize}
 		
	\subsection{Wann soll bestellt werden?}
 	
 		\begin{itemize}
 			\item \textbf{Ziele:} Termingerechte Lieferung, geringer Einstandspreis
 			\item \textbf{Zeitplanung} abhängig von: Warenbestand, 
 		\end{itemize}
 		
 	\subsection{Zu welchem Preis soll bestellt werden?}
 	
 		\begin{itemize}
 			\item \textbf{Ziele:} Möglichst günstiger EK \footnote{Einkaufspreis}
 			\item \textbf{Preisplanung} abhängig von: Budget, Kosten, Marktlage
 		\end{itemize}
 		
 	\subsection{Wo soll bestellt werden?}
 	
 		\begin{itemize}
 			\item \textbf{Ziele:}Aufbau eines zuverlässigen Lieferantenstamms
 			\item \textbf{Bezugsquellenermittlung} abhängig von: Angebot, Reputation
 		\end{itemize}
\end{document}
