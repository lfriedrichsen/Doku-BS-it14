%% cheat sheet
% Dokumentformatierung
	% \\				< Zeilenumbruch
	% \par				< Absatz mit Einrückung
	% \\[0.5cm]			< Absatz mit Abstand
	% \newpage			< Neue Seite
	% \part 			< Teil
	% \chapter 			< Kapitel
	% \section 			< Unterüberschrift
	% \subsection 		< Unterunterüberschrift

% Textformatierung
	% \glqq{}foo\{}		< Anführungszeichen
	% \textit{foo}		< Kursiv
	% \textbf{foo}		< Fett
	% \underline{foo}	< Unterstrichsen

% ----------------------------------------------------------------------

% Dokumentenklasse / Definitionen
\documentclass[a4paper,11pt]{scrartcl}	% Format des Dokuments
\usepackage[utf8]{inputenc}				% Interpreter
\usepackage[ngerman] {babel}			% Definition des Sprachraums
\usepackage[T1] {fontenc}				% Schriftart
\usepackage {graphicx}					% Erlaubt das einfügen von Bildern

% Kopf- Fußzeile
\usepackage{fancyhdr}
\pagestyle{fancy}
\lhead{LF 1 / Hi}
\chead{Thema: Grundlagen des wirtschaftens}
\rhead{Datum: 31.10.2014}
\lfoot{\thepage}
\cfoot{}
\rfoot{Lars Hendrik Friedrichsen - Version 1.0 - \today}


% Header
\title{Beschaffungsablauf}			
\author{Lars Friedrichsen}
\date{\today}

% Dokument / Printing 
\begin{document}

\section{Wirtsschaftswissenschaften}

Im speziellen beschäftigen wir uns zunächst mit dem Gegenstand und der Abgrenzung des Faches BWL.
Somit beschäftigen wir uns zunächst mit den verschiedenen Disziplinen der \textbf{Wirtschaftswissenschaften}.
Zu ihnen gehören folgende Unterdisziplinen:

	\begin{itemize}
		\item Recht
		\item Mathematik
		\item Informatik
		\item Psychologie
		\item Soziologie
		\item \dots
	\end{itemize}


Die Wirtschaftswissenschaften sind in zwei große Bereiche unterteilt. Diese beiden Bereiche werden im folgenden näher betrachtet.

	\subsection{BWL (Betriebswirtschaftslehre)}
	
	Der Gegenstand der der \textbf{BWL} ist der einzelne Betrieb bzw. die einzelne Unternehmung.
	
		\begin{itemize}
			\item Einkauf, Fertigung, Absatz
			\item Lagerhaltung
			\item Finanzierung
			\item Kalkulation (-> Kostenrechnung)
			\item Buchführung
			\item Organisation
			\item \dots
		\end{itemize}





	\subsection{VWL (Volkswirtschaftslehre)}
	
	Die \textbf{VWL} beschäftigt sich mit der gesamten Wirtschaft, z.B. der eines Landes.
	
		\begin{itemize}
			\item Leistung der gesamten Wirtschaft (-> Bruttosozialprodukt)
			\item Ländervergleich
			\item Wirtschaftssysteme
			\item Rolle des Staates in der Wirtschaft (-> Wirtschaftspolitik)
			\item \dots
		\end{itemize}
	
\section{Die drei Wirtschaftssektoren}

Die Wirtschaft ist in drei verscheiden Bereiche unterteilt. Jedem Sektor ist eine bestimmte Rolle zugewiesen.
Diese Aufteilung folgt dem Prinzip der Wirtschaftswissenschaften, nämlich der vereinfachten Darstellung von 
komplexen Abläufen.\\[0.5cm]
Im folgenden sind die verschieden Sektoren und deren Rolle zusammengefasst. \\

	\begin{tabular}{|l|l|l|}
	\hline
		\textbf{primärer Sektor}	&	\textbf{sekundärer Sektor}	&	\textbf{tertiärer Sektor} \\ \hline
		\= Urproduktion				&	\= Produktion				&	\= Dienstleistungen \\ \hline
		Land- u. Forstwirtschaft	& 	Industrie, Handwerk			&	Handel, Banken, Transport \\ \hline
	\end{tabular}

	\subsection{Entwicklung der Beschäftigung in den verschiedenen Sektoren}
	
	Es ist zu beobachten, dass es innerhalb der letzten 100 Jahre zu einer stetigen Verschiebung der Beschäftigungsverhältnisse gekommen ist.
	Dazu sei gesagt, dass sich die Verschiebung zum \textbf{tertiären} Sektor hinbewegt. Als Begründung seien die
	folgenden Punkte genannt:
		
		\begin{itemize}
			\item Automatisierung durch technischen Fortschritt
			\item Globalisierung -> Auslagerung der Arbeitsplätze			
		\end{itemize}		 

	
	
\end{document}
