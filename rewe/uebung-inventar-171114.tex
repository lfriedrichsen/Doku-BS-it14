%% cheat sheet
% Dokumentformatierung
	% \\				< Zeilenumbruch
	% \par				< Absatz mit Einrückung
	% \\[0.5cm]			< Absatz mit Abstand
	% \newpage			< Neue Seite
	% \part 			< Teil
	% \chapter 			< Kapitel
	% \section 			< Unterüberschrift
	% \subsection 		< Unterunterüberschrift

% Textformatierung
	% \glqq{}foo\{}		< Anführungszeichen
	% \textit{foo}		< Kursiv
	% \textbf{foo}		< Fett
	% \underline{foo}	< Unterstrichsen

% ----------------------------------------------------------------------

% Dokumentenklasse / Definitionen
\documentclass[a4paper,11pt]{scrartcl}	% Format des Dokuments
\usepackage[utf8]{inputenc}				% Interpreter
\usepackage[ngerman] {babel}			% Definition des Sprachraums
\usepackage[T1] {fontenc}				% Schriftart
\usepackage{graphicx}					% Erlaubt das einfügen von Bildern

% Kopf- Fußzeile
\usepackage{fancyhdr}
\pagestyle{fancy}
\lhead{LF 11 / Pr}
\chead{Thema: Inventur / Inventar}
\rhead{Datum: 17.11.2014}
\lfoot{Lars Hendrik Friedrichsen - Version 1.0 - \today}
\cfoot{}
\rfoot{\thepage}


% Header
\title{Einführung ReWe}			
\author{Lars Friedrichsen}
\date{\today}

% Dokument / Printing 
\begin{document}

\section{Übungsaufgaben zur Inventur}
	
	\subsection{Stellen Sie fest, welche Inventurart in den folgenden Fällen jeweils gewählt wurde.}

	\begin{enumerate}
		\item Die Inhaberin eines IT-Unternehmens hat als Bilanzstichtag den 31.08. eines jeden Jahres gewählt.
			An diesem Tag führt sie auch jeweils die Inventur durch, weil in den verkaufsschwächeren Sommermonaten
			die Warenbestände besonders niedrig sind.\par
			\underline{Antwort:} Es handelt sich um die \textbf{Stichtagsinventur}.
		\item In einem Elektronik-Fachmarkt werden alle Wareneingänge sorgfältig per Computer erfasst. Jeder
			Warenverkauf wird nicht nur in der Kasse registriert, sondern veranlasst gleichzeitig den Computer,
			den Warenbestand entsprechend zu korrigieren. Es kann also immer der aktuelle Warenbestand genannt
			werden. Die Inventur wird abteilungsweise über das Jahr verteilt, um den Geschäftsablauf nicht
			zu behindern. Bilanzstichtag ist jeweils der 31.12 des Jahres.\par
			\underline{Antwort:} Es handelt sich um die \textbf{permanente Inventur}.
		\item In einem anderen Fachmarkt, wird die am 31.01 gemacht, obwohl der Bilanzstichtag der 31.12. Man hat dieses
			Verfahren gewählt, da die Bestände nach dem Weihnachtsgeschäft besonders niedrig sind.\par
			\underline{Antwort:} Es handelt sich um die \textbf{verlegte Inventur}.
	\end{enumerate}

	\subsection{Ordnen Sie folgende Inventarpositionen nach Vermögen und Schulden und sortieren Sie diese nach Liquidität,
		bzw. Fälligkeit.}

		\begin{tabular}{|l|l|}
			\hline
			\textbf{Vermögen} 	& 	\textbf{Schulden}\\ \hline
			Grundstücke 		& 	Eigenkapital\\ \hline
			Gebäude 		& 	Darlehn\\ \hline
			Fuhrpark 		& 	Verbindlichkeiten\\ \hline
			BGA 			& 	\ \\ \hline
			Warenvorräte 		& 	\ \\ \hline
			Forderungen 		& 	\ \\ \hline
			Kasse 			& 	\ \\ \hline
			Bank 			& 	\ \\ \hline
		\end{tabular}

	\subsection{Begünden Sie warum Forderungen liquider sind als Warenbestände}

	Forderungen sind liquider, da die Ware schon verkauft ist. Das heißt, die Ware ist nicht mehr im Lager gebunden. 
	Die Forderung stellt einen direkten Anspruch auf das Geld vom Kunden dar.

\newpage

	\subsection{Stellen Sie eine Wertrückrechnung nach den Vorschriften der zeitlich verlegten Inventur auf}

		\begin{itemize}
			\item Bezugspreis: EUR 58,- 
			\item Bestand lt. Inbentur 16.02: 80 Stk.
			\item Bestandsänderungen zwischen 02.01 und 16.02:
				\begin{itemize}
					\item Zugänge: 15 Stk.
					\item verkäufe: 25 Stk
				\end{itemize}
		\end{itemize}
			
		\begin{enumerate}
			\item \textbf{Wie hoch war der Inventurbestand und -wert am 31.12?}\par
				\underline{Anwort:} \textbf{Bestand:} $80 + 25 - 15 = 90$\\
				Der Inventurbestand am 31.12. war 90 Stk.\\
				\textbf{Wert:} $90 * 58 = 5.520$\\
				Der Inventurwert beträgt EUR 5.520,-
			\item \textbf{Im September des letzten Jahres betrug der Bezugspreis EUR 53,95. Um wie viel Prozent
				hat sich der Bezugspreis verändert?}\par
				\underline{Antwort:} \textbf{Preisänderung:} $58 / 53,95 - 1 = 7,51\%$\\
				Der Preis hat sich um 7,51\% verändert.
		\end{enumerate}

	\subsection{Ermitteln sie für das folgende Unternehmen die Inventurbestände zum Abschlussstichtag!}

		\begin{enumerate}
			\item Die zeitlich verlegte Inventur am 05.10. hat einen Bestand von EUR 96.500,00 ergeben. In der Zeit
				bis zum Abschlussstichtag am 31.12 wurden Waren im Wert von EUR 36.000,00 eingekauft.
				Im gleichen Zeitraum wurden Waren im Wert von EUR 79.000,00 verkauft.\par
				\underline{Antwort:} \textbf{Bestandsänderungen:} $96.500 + 36.000 - 79.000 = 53.500$\\
				Der Bestand hat sich wertmäßig auf EUR 53.500 geändert.
			\item Die zeitlich verlegte Inventur am 15.02 im Folgejahr hat einen wertmäßigen Bestand von EUR 81.000
				ergeben. In der Zeit vom 02.01 bis zum 15.02 wurden folgende Bestandsänderungen festgehalten:
					\begin{itemize}
						\item Einkäufe: EUR 21.300
						\item Verkäufe: EUR 36.000
					\end{itemize}
				\underline{Antwort:} \textbf{Bestandänderungen:} $81.000 - 21.300 + 36.000 = 95.700$\\
				Der Bestand hat sich auf EUR 95.700 geändert.
		\end{enumerate}
	
	\subsection{Am Geschäftsjahresende wurde ein Eigenkapital von EUR 799.000,00 ermittelt.	Wie hoch ist der Erfolg der
	Unternehmung zum 31.12. des zweiten Geschäftsjahres, wenn folgende Daten zu Grunde liegen?}

		\begin{itemize}
			\item EK zum Ende des zweiten Geschäftsjahres: EUR 903.00,00
			\item Privateinlagen: EUR 21.000,00
			\item Privatentnahme: EUR 60.000,00
		\end{itemize}

	\underline{Antwort:} Der Erfolg nach dem zweiten Geschäftsjahr beträgt EUR 104.000,00 (Gewinn) und ergibt sich aus
	folgender Rechnung:\par
	$903.000 - 799.00 = 104.000$\\
	Daraus ergibt sich ein Gewinn von EUR 104.000,00.

\end{document}
	
