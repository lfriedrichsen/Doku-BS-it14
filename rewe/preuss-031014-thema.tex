%% cheat sheet
% Dokumentformatierung
	% \\				< Zeilenumbruch
	% \par				< Absatz mit Einrückung
	% \\[0.5cm]			< Absatz mit Abstand
	% \newpage			< Neue Seite
	% \part 			< Teil
	% \chapter 			< Kapitel
	% \section 			< Unterüberschrift
	% \subsection 		< Unterunterüberschrift

% Textformatierung
	% \glqq{}foo\{}		< Anführungszeichen
	% \textit{foo}		< Kursiv
	% \textbf{foo}		< Fett
	% \underline{foo}	< Unterstrichsen

% ----------------------------------------------------------------------

% Dokumentenklasse / Definitionen
\documentclass[a4paper,11pt]{scrartcl}	% Format des Dokuments
\usepackage[utf8]{inputenc}				% Interpreter
\usepackage[ngerman] {babel}			% Definition des Sprachraums
\usepackage[T1] {fontenc}				% Schriftart
\usepackage {graphicx}					% Erlaubt das einfügen von Bildern

% Kopf- Fußzeile
\usepackage{fancyhdr}
\pagestyle{fancy}
\lhead{LF 11 / Pr}
\chead{Thema: Einführung ReWe}
\rhead{Datum: 03.11.2014}
\lfoot{\thepage}
\cfoot{}
\rfoot{Lars Hendrik Friedrichsen - Version 1.0 - \today}


% Header
\title{Einführung ReWe}			
\author{Lars Friedrichsen}
\date{\today}

% Dokument / Printing 
\begin{document}

\section{Allgemeine Informationen}
Andreas verkürzt seine Ausbildung und bittet uns die Mitschriften aus der Unterstufe an ihn weiterzureichen.
Er wiederum versorgt uns mit den Informationen aus der Mittelstufe. Wer Interesse hat ihn zu ubterstützen, kann sich
per Mail an ihn wenden.\\[0.5cm] 
\textbf{mail@andreas-elbert.de}


\section{Aufgaben des Rechnungswesen}
Das Rechnungswesen ist in vier Unterbereiche aufgeteilt (Finanzbuchhaltung, Kosten- u. Leistungsrechnung, Statistik und Planungsrechnung)

	\subsection{Finanzbuchhaltung}
			\begin{itemize}
			\item \textbf{Dokumentation:} Alle (vermögenswirksamen) Geschäftsvorfälle werden erfasst
			\item \textbf{Information / Rechenschaftslegung:} Der Jahresabschluss informiert die Geschäftsleitung und
			Außenstehende (z.B. Kreditinstitute, Lieferanten) über:
				\begin{itemize}
					\item Die Zusammensetzung von Vermögen und Schulden
					\item Den  Erfolg (Gewinn oder Verlust)
					\item Art und Höhe der Aufwendungen und Erträge im Einzelnen
				\end{itemize}
			\item \textbf{Besteuerungsgrundlage:} Der Jahresabschluss ist Grundlage für die Unternehmensbesteuerung 
			(Est, KöSt, Ust, GewerbeSt).
			\item \textbf{Gläubigerschutz:} Kreditgeber können sich ein Bild über die wirtschaftliche / finanzielle
			Lage des Unternehmens machen.
			\item \textbf{Beweismittel:} Bücher und Aufzeichnungen können bei Rechtsstreitigkeiten als Beweismittel
			herangezogen werden. 
		\end{itemize}			 

	\subsection{Kosten- u. Leistungsrechnung}
		\begin{itemize}
			\item \textbf{Betriebsbuchhaltung:} Zuordnung der Kosten zu den einzelnen Abteilungen (Kostenstellen)
			\item \textbf{Selbstkostenrechnung:} Kalkulationsgrundlage für die Ermittlung der Verkaufspreise 
		\end{itemize}
		
\newpage
	
	\subsection{Statistik}
	Aufbereitung und Auswertung der Zahlen aus der Buchführung und Kosten- u. Leistungsrechnung \\[0.5cm]
	Die Statistik dient der Überwachung (Controlling) des Unternehmens. Die aufbereiteten Zahlen werden mit denen aus
	\textbf{früheren} Perioden des eigenen Unternehmens oder mit \textbf{anderen} Unternehmen verglichen.
	(Perioden- u. Unternehmensvergleich)
	
	\subsection{Planungsrechnung}
	Ermittlung von Planzahlen auf der Grundlage der Zahlen aus den übrigen Gebieten des Rechnungswesens.
	
		\begin{itemize}
			\item Zukünftige Ein - und Ausgaben werden geschätzt und als Richtwert festgesetzt
			\item Vergleich der geplanten mit den tatsächlichen Zahlen
			\item Analyse der Ursachen bei Abweichungen
		\end{itemize}
		
	\subsection{Merksatz}
	Das betriebliche Rechnungswesen kann damit als \textbf{Informations- Steuerungs- und Kontrollsystem} bezeichnet werden.
	
\section{Inventur und Inventar}

	\subsection{Die Inventur}
		\begin{itemize}
			\item \textbf{Begriff:} Unter der Inventur versteht man die Bestandsaufnahme aller Vermögensgegenstände und Schulden.
			\item \textbf{Zeitpunkt:}
				\begin{itemize}
					\item zum Ende des Geschäftsjahres oder
					\item bei Gründung, Übernahme, Verkauf oder Auflösung des Unternehmens
				\end{itemize}				 
			\item \textbf{Arten:} 
				\begin{enumerate}
					\item \textbf{Stichtagsinventur}, d.h. innerhalb von 10 Tagen vor oder nach dem Ende des Geschäftsjahres
					\item \textbf{Verlegte Inventur}, d.h. bis zu drei Monate vor oder bis zu zwei Monate nach dem Ende des Geschäftsjahres
				\end{enumerate}
		\end{itemize}
		
	\subsection{Das Inventar}
	
	

\end{document}
	