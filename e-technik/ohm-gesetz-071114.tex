%% cheat sheet
% Dokumentformatierung
	% \\					< Zeilenumbruch
	% \par				< Absatz mit Einrückung
	% \\[0.5cm]			< Absatz mit Abstand
	% \newpage			< Neue Seite
	% \part 				< Teil
	% \chapter 			< Kapitel
	% \section 			< Unterüberschrift
	% \subsection 		< Unterunterüberschrift

% Textformatierung
	% \glqq{}foo\{}		< Anführungszeichen
	% \textit{foo}		< Kursiv
	% \textbf{foo}		< Fett
	% \underline{foo}	< Unterstrichsen

% ----------------------------------------------------------------------

% Dokumentenklasse / Definitionen
\documentclass[a4paper,11pt]{scrartcl}	% Format des Dokuments
\usepackage[utf8]{inputenc}				% Interpreter
\usepackage[ngerman] {babel}			% Definition des Sprachraums
\usepackage[T1] {fontenc}				% Schriftart
\usepackage {graphicx}					% Erlaubt das einfügen von Bildern

% Kopf- Fußzeile
\usepackage{fancyhdr}
\pagestyle{fancy}
\lhead{Elektrotechnik}
\chead{Thema: Das ohmsche Gesetz}
\rhead{Datum: 07.11.2014}
\lfoot{\thepage}
\cfoot{}
\rfoot{Lars Hendrik Friedrichsen - Version 1.0 - \today}


% Header
\title{Das ohmsche Gesetz}			
\author{Lars Friedrichsen}
\date{\today}

% Dokument / Printing 
\begin{document}

\section{Das Ohmsche Gesetz}
Das Ohmsche Gesetz besagt, dass die Stromstärke \textbf{I} in einem Leiter und die Spannung \textbf{U} zwischen den Enden des Leiters direkt proportional sind. Die mathematische Darstellung sieht wie folgt aus:\par
$U = R * I$

	\subsection{Übungsaufgaben zum ohmschen Gesetz}
		\begin{enumerate}
			\item \textbf{Woran läßst sich erkennen, welcher Widerstand der größste ist?}\par
			R1 ist der größte Widerstand, da die Steigung der Kurve am geringsten ist.
			\item \textbf{Ermitteln sie die Größen der einzelnen Widerstände R1, R2 und R3}\par
				\begin{tabular}{|l|l|l|l|}
					\hline
							&	R1		&	R2	&	R3\\ \hline
					U in V	&	50		&	50	&	20\\ \hline
					I in A	&	0,2		&	0,4	&	0,4\\ \hline
				\end{tabular}\\[0.5cm]
			Es gilt das ohmsche Gesetz: $U = R * I$\\
			Daraus folgt: $R = U / I$\\[0.5cm]
			Somit ergeben sich folgende Ergebnisse:
				\begin{enumerate}
					\item $R1 = 250\ \Omega$
					\item $R2 = 125\ \Omega$
					\item $R3 = 50\ \Omega$				
				\end{enumerate}							
			\item \textbf{Welche Stromänderung tritt an R2 auf, wenn die Spannung ($U = 50V$) um 20\% erhöht wird?
			Geben Sie die die prozentuale Änderung der Stromstärke an!}\par
			Da $U$ und $I$ proportional zueinander sind, verändert sich auch der Strom um 20\%.
			\item \textbf{Wodurch lässt sich die Stromstärke an R3 von 0,8A auf 0,5A senken?}\par
			Die Stromstärke lässt sich auf 0,5A senken, wenn man eine Spannung von \textbf{25V} anlegt.
			
		\end{enumerate}
	
\end{document}
