%% cheat sheet
% Dokumentformatierung
	% \\					< Zeilenumbruch
	% \par				< Absatz mit Einrückung
	% \\[0.5cm]			< Absatz mit Abstand
	% \newpage			< Neue Seite
	% \part 				< Teil
	% \chapter 			< Kapitel
	% \section 			< Unterüberschrift
	% \subsection 		< Unterunterüberschrift

% Textformatierung
	% \glqq{}foo\{}		< Anführungszeichen
	% \textit{foo}		< Kursiv
	% \textbf{foo}		< Fett
	% \underline{foo}	< Unterstrichsen

% ----------------------------------------------------------------------

% Dokumentenklasse / Definitionen
\documentclass[a4paper,11pt]{scrartcl}	% Format des Dokuments
\usepackage[utf8]{inputenc}				% Interpreter
\usepackage[ngerman] {babel}			% Definition des Sprachraums
\usepackage[T1] {fontenc}				% Schriftart
\usepackage {graphicx}					% Erlaubt das einfügen von Bildern

% Kopf- Fußzeile
\usepackage{fancyhdr}
\pagestyle{fancy}
\lhead{Elektrotechnik}
\chead{Thema: Der elektrische Strom}
\rhead{Datum: 30.10.2014}
\lfoot{}
\cfoot{}
\rfoot{Lars Hendrik Friedrichsen - Version 1.0 - \today}


% Header
\title{Beschaffungsablauf}			
\author{Lars Friedrichsen}
\date{\today}

% Dokument / Printing 
\begin{document}

\section{Wirkung des elektrischen Stromes}


	\subsection{Wirkungsarten und Beispiele}

		\begin{enumerate}
			\item physiologische Wirkung (Weidezaun, Medizin)
			\item Magnetismus (Lautsprecher, Festplatte, Lüftermotoren)
			\item Wärmewirkung (Laserdrucker, Grafikkarte)
			\item Lichtwirkung (Lampen)
			\item chemische Wirkung (Batterien, Akkus)
		\end{enumerate}
			
	\subsection{Bezeichnung eines Diagramms}
	
	Ein Diagramm ist in X- und Y-Achse dargestellt. Die X-Achse ist die Abszisse und die Y-Achse die Ordinate.
		\begin{itemize}
			\item Auf der Abszisse wird immer die \textbf{veränderliche} Größe angegeben.
			\item Auf der Ordinate wird die davon \textbf{abhängige} Größe dargestellt.
		\end{itemize}
		
	\subsection{Messung von Spannung und Stromstärke}
	
	Die Spannung wird immer \textbf{parallel} zum Verbraucher gemessen; die Stromstärke hingegen in \textbf{reihe}.
		
		\begin{enumerate}
			\item{Die richtige Spannungsart muss ausgewählt werden (AC/DC).}
    		\item{Bei Gleichspannung (DC) muss die Polarität beachtet werden.}
    		\item{richtige Messbereich muss eingestellt werden.}
    		\item{Bei einem unbekannten Messwert muss der größte Messbereich eingestellt und langsam in die niedrigeren Messbereich geschaltet werden.}
    		\item{Der Messbereich muss möglichst so eingestellt werden, dass der Zeigerausschlag im letzten Drittel abgelesen werden kann.}
		\end{enumerate}
		
\end{document}
