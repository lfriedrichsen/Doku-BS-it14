%% cheat sheet
% Dokumentformatierung
	% \\					< Zeilenumbruch
	% \par				< Absatz mit Einrückung
	% \\[0.5cm]			< Absatz mit Abstand
	% \newpage			< Neue Seite
	% \part 				< Teil
	% \chapter 			< Kapitel
	% \section 			< Unterüberschrift
	% \subsection 		< Unterunterüberschrift

% Textformatierung
	% \glqq{}foo\{}		< Anführungszeichen
	% \textit{foo}		< Kursiv
	% \textbf{foo}		< Fett
	% \underline{foo}	< Unterstrichsen

% ----------------------------------------------------------------------

% Dokumentenklasse / Definitionen
\documentclass[a4paper,11pt]{scrartcl}	% Format des Dokuments
\usepackage[utf8]{inputenc}				% Interpreter
\usepackage[ngerman] {babel}			% Definition des Sprachraums
\usepackage[T1] {fontenc}				% Schriftart
\usepackage {graphicx}					% Erlaubt das einfügen von Bildern

% Kopf- Fußzeile
\usepackage{fancyhdr}
\pagestyle{fancy}
\lhead{Elektrotechnik}
\chead{Thema: Der elektrische Strom}
\rhead{Datum: 31.10.2014}
\lfoot{\thepage}
\cfoot{}
\rfoot{Lars Hendrik Friedrichsen - Version 1.0 - \today}


% Header
\title{Beschaffungsablauf}			
\author{Lars Friedrichsen}
\date{\today}

% Dokument / Printing 
\begin{document}

\section{Basiswissen Elektrotechnik}


	\subsection{Klausurtermin}
	
	\textbf{Freitag; KW 48; 28.11.2014}
	
	\subsection{Der einfache elektrische Stromkreis}
	
	\textbf{Merke:}
	
		\begin{itemize}
			\item Damit ein elektrischer Strom fließt, muss der Stromkreis geschlossen sein
			\item Der Schaltplan zeigt nur die Funktion der Schaltung und nicht die technische Umsetzung
		\end{itemize}
	
	Zu einem einfachen Stromkreis gehören folgende Komponenten:
	
		\begin{enumerate}
			\item Energiequelle
			\item Hin- und Rückleiter
			\item Schalter
			\item \glqq Verbraucher\grqq
		\end{enumerate}

	\subsection{Die elektrische Spannung}
	
	Unter der elektrischen Spannung versteht man das Bestreben unterschiedlicher Ladungen, sich auszugleichen. \\[0.5cm]
	Formelzeichen: \textbf{U}, in der Einheit \textbf{V} (Volt). \\[0.5cm]
	Man unterscheidet zwischen zwei Spannungsquellen, der \textbf{Gleichspannung} und der \textbf{Wechselspannung}.
	Bei einer Gleichspannungsquelle werden die Anschlüsse als Plus- und Minuspol bezeichnet. 
	Bei einer Wechselspannungsquelle hingegen spricht man von einem \textbf{Außenleiter (L1) und Neutralleiter (N)}.
	
	\subsection{Der elektrische Strom}
	
	Unter dem elektrischen Strom versteht man den die gerichtete Bewegung von Elektronen. In der Praxis wird die technische
	Fließrichtung (von plus nach minus) verwendet. Diese kann auch mit einem Pfeil in einem Schaltplan eingezeichnet werden.
	Die Stromstärke \textbf{I} wird in Ampère gemessen.	
			
\end{document}
