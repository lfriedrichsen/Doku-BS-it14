%% cheat sheet
% Dokumentformatierung
	% \\				< Zeilenumbruch
	% \par				< Absatz mit Einrückung
	% \\[0.5cm]			< Absatz mit Abstand
	% \newpage			< Neue Seite
	% \part 			< Teil
	% \chapter 			< Kapitel
	% \section 			< Unterüberschrift
	% \subsection 		< Unterunterüberschrift

% Textformatierung
	% \glqq{}foo\{}		< Anführungszeichen
	% \textit{foo}		< Kursiv
	% \textbf{foo}		< Fett
	% \underline{foo}	< Unterstrichsen

% ----------------------------------------------------------------------

% Dokumentenklasse / Definitionen
\documentclass[a4paper,11pt]{scrartcl}	% Format des Dokuments
\usepackage[utf8]{inputenc}				% Interpreter
\usepackage[ngerman] {babel}			% Definition des Sprachraums
\usepackage[T1] {fontenc}				% Schriftart
\usepackage {graphicx}					% Erlaubt das einfügen von Bildern

% Kopf- Fußzeile
\usepackage{fancyhdr}
\pagestyle{fancy}
\lhead{LF 8 / Hiss}
\chead{Thema: Der Ausbildungsvertrag}
\rhead{Datum: 03.11.2014}
\lfoot{\thepage}
\cfoot{}
\rfoot{Lars Hendrik Friedrichsen - Version 1.0 - \today}


% Header
\title{Thema}			
\author{Lars Friedrichsen}
\date{\today}

% Dokument / Printing 
\begin{document}

\section{Der Ausbildungsvertrag}

	\subsection{Rechte des Auszubildenden}
		\begin{itemize}
			\item \textbf{Vergütung:} Recht auf eine angemessene Vergütung
			\item \textbf{Freistellung:} Freistellung für die Berufsschule
			\item \textbf{Ausbildungsziel:} Keine Tätigkeiten, die nicht zur Ausbildung gehören 
			\item \textbf{Kündigungsrecht:} Auszubildende haben ein Kündigungsrecht mit einer Frist von \textbf{4 Wochen}
			\item \textbf{Zeugnis:} Der Auszubildende hat das Recht auf ein Ausbildungszeugnis
			\item \textbf{Vertretung:} Unter Umständen muss eine Auszubildendenvertretung gebildet werden
		\end{itemize}
	
	\subsection{Pflichten des Auszubildenden}
		\begin{itemize}
			\item \textbf{Lernpflicht:} Der Auszubildende ist verpflichtet die Ausbildung erfolgreich abschließen zu wollen
			\item \textbf{Sorgfaltspflicht:} Aufgaben sollen ordentlich und sorgfältig erfüllt werden 
			\item \textbf{Teilnahmepflicht:} Der Besuch der Berufsschule ist verbindlich
			\item \textbf{Anweisungen:} Der Auszubildende hat die Anweisungen des Ausbilders zu befolgen
			\item \textbf{Betriebsordnung:} Die Betriebsordnung ist einzuhalten (z.B. Schutzkleidung)
			\item \textbf{Bewahrungspflicht:} Der Auszubildende hat sorgsam mit den Arbeitsmitteln umzugehen
			\item \textbf{Schweigepflicht:} Betriebsgeheimnisse sind zu wahren
			\item \textbf{Krankmeldung:} Es ist eine Krankheitsmeldung abzugeben
		\end{itemize}
	
\section{Aufgaben}

\begin{enumerate}
	\item Notieren Sie das Ziel der Berufsausbildung! (§1 BBiG)\par
	
	\item Welches sind die zwei Lernorte der beruflichen Ausbildung? (§2 BBiG)\par
	Die Ausbildung findet in der \textbf{Berufsschule} und im \textbf{Ausbildungsbetrieb} statt.
	\item Wie wird dieses Ausbildungssystem mit den zwei Lernorten genannt?\par
	Es handelt sich um das \textbf{duale Ausbildungssystem}.
	\item Was besagt das BBIG zur Ausbildungsdauer? (§5 (1) BBiG)\par
	Die Ausbildungsdauer bewegt sich zwischen \textbf{zwei und drei Jahren} 
	\item Damit ein Ausbildungsberuf auch bundesweit anerkannt ist, muss er auch inhaltlich geregelt sein. Wie heißt diese Ordnung?\par
	Es handelt sich um die \textbf{Ausbildungsordnung}.
	\item Was ist der Unterschied zwischen \glqq Ausbildungsberufsbild\grqq und \glqq Ausbildungsrahmenplan\grqq ? (§5 (1) BBiG)\par
	\item Wer muss den Antrag stellen, wenn die Ausbildungszeit verkürzt werden soll und wo muss dieser gestellt werden? (§8 (1) BBiG)\par
	Der Antrag muss vom \textbf{Auszubildenden und Ausbildenden} bei der \textbf{IHK} gestellt werden.
	\item Welche Form benötigt der Ausbildungsvertrag und wann ist dies spätestens zu erfüllen? (§11 BBiG)\par
	Der Vertrag ist an die \textbf{Schriftform} gebunden und muss vor Ausbildungsbeginn vorhanden sein.
	\item Azubi Frank erzählt seinem Freund, in seinem Ausbildungsvertrag steht geschrieben: Arbeitszeit: 
	je nach Bedarf. Sein Ausbilder meint, das könne mal 5 Stunden sein, aber auch mal 11. Ist diese individuelle
	Regelung erlaubt? (§11 (1) Nr. 4)\par
	Diese Regelung ist nicht zulässig, da die Arbeitszeit im Ausbildungsvertrag geregelt ist.
	\item Frank ist begeistert, dass er jetzt auch weiß, wie viel Umsatz und Gewinn ein Einzelhandelsunternehmen 
	macht. In der Kneipe erzählt er, dass sein Betrieb letztes Jahr EUR 50,000 Miese gemacht hat. Dieses Jahr laufe
	es aber besser, Sie machten einen Umsatz von EUR 10,000 am Tag. 
	\textbf{Gegen welche Pflicht hat er verstoßen?}\par
	Er hat gegen die Schweigepflicht verstoßen §13 Nr. 6
	\item Welche Angaben muss ein Ausbildungszeugnis enthalten? (§ 16 BBiG)\par
	Der Ausbildungsvertrag muss die Dauer, Art, Zeit und die beruflichen Fertigkeiten regeln.
	\item Ein Auszubildender hat Anspruch auf eine angemessene Vergütung (§17 BBiG). Welche Ansprüche hat er, wenn
	er regelmäßig mehr als die vereinbarte Zeit arbeitet? (§17 (3)\par
	Diese Zeit ist besonders zu vergüten oder durch entsprechende Freizeit auszugleichen.
	\item Was geschieht, wenn der Auszubildende seine Abschlussprüfung nicht besteht? (§21 BBiG)\par
	Er kann an der nächsten Prüfung teilnehmen.
	\item Ein Kollege vom Auszubildenden Frank wurde mit den Worten \glqq Hau ab, lass dich nie wieder blicken!\grqq
	gekündigt. Ist die Kündigung wirksam? (§22 BBiG)\par
	Die Kündigung ist nicht wirksam, da eine Kündigung der Schriftform bedarf.
	\item Nach bestandener Abschlussprüfung geht Karin in den Betrieb und arbeitet wie gewohnt weiter.
	Besteht nun ein normales Arbeitsverhältnis oder muss sie neu mit ihrer Chefin verhandeln? (§24 BBiG)\par
	Es besteht ein normales Arbeitsverhältnis, demnach muss nicht erneut verhandelt werden.
		
\end{enumerate}


\end{document}
	